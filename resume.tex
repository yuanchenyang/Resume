\documentclass[9pt]{article}
\usepackage{fullpage}
\usepackage{amsmath}
\usepackage{amssymb}
\usepackage[usenames]{color}
\usepackage{hyperref}
\usepackage{multicol}

\leftmargin=0.25in
\oddsidemargin=0.25in
\textwidth=6.0in
\topmargin=-0.25in
\textheight=9.25in

\raggedright

\pagenumbering{arabic}

\def\bull{\vrule height 0.8ex width .7ex depth -.1ex }
% DEFINITIONS FOR RESUME

\newenvironment{changemargin}[2]{%
  \begin{list}{}{%
      \setlength{\topsep}{0pt}%
      \setlength{\leftmargin}{#1}%
      \setlength{\rightmargin}{#2}%
      \setlength{\listparindent}{\parindent}%
      \setlength{\itemindent}{\parindent}%
      \setlength{\parsep}{\parskip}%
    }%
  \item[]}{\end{list}
}

\newcommand{\lineover}{
  \begin{changemargin}{-0.05in}{-0.05in}
    \vspace*{-8pt}
    \hrulefill \\
    \vspace*{-2pt}
  \end{changemargin}
}

\newcommand{\header}[1]{
  \begin{changemargin}{-0.5in}{-0.5in}
    \scshape{#1}\\
    \lineover
  \end{changemargin}
}

\newcommand{\contact}[4]{
  \begin{changemargin}{-0.5in}{-0.5in}
    \begin{center}
      {\Large \scshape {#1}}\\ \smallskip
      {#2}\\ \smallskip
      {#3}\\ \smallskip
      {#4}\smallskip
    \end{center}
  \end{changemargin}
}

\newenvironment{body} {
  \vspace*{-16pt}
  \begin{changemargin}{-0.25in}{-0.5in}
  }
  {\end{changemargin}
}

\newcommand{\school}[4]{
  \textbf{#1} \hfill \emph{#2\\}
  #3\\
  #4\\
}

% END RESUME DEFINITIONS

\begin{document}

%%%%%%%%%%%%%%%%%%%%%%%%%%%%%%%%%%%%%%%%%%%%%%%%%%%%%%%%%%%%%%%%%%%%%%%%%%%%%%%%
% Name
\contact{Chenyang Yuan}{\href{mailto:yuanchenyang@gmail.com}{yuanchenyang@gmail.com}}{ \url{http://www.github.com/yuanchenyang} \quad \url{http://www.chenyang.co}}

%%%%%%%%%%%%%%%%%%%%%%%%%%%%%%%%%%%%%%%%%%%%%%%%%%%%%%%%%%%%%%%%%%%%%%%%%%%%%%%%
% Education
\header{Education}

\begin{body}
  \vspace{14pt}
  \textbf{Double Major in Computer Science and Physics} \hfill Expected Graduation: 2016 \\
  \emph{The University of Berkeley at California}, Berkeley, CA{} \hfill GPA: 3.939 (Technical: 4.00)\\
\end{body}

\smallskip

%%%%%%%%%%%%%%%%%%%%%%%%%%%%%%%%%%%%%%%%%%%%%%%%%%%%%%%%%%%%%%%%%%%%%%%%%%%%%%%%
% Skills
\header{Technical Skills}

\begin{body}
  \vspace{14pt}
  \textbf{Proficient in} Python, Haskell, Javascript, Java,  C, \LaTeX, Emacs \\
  \textbf{Experience in} Rust, Scheme, jQuery, d3, HTML, Hadoop, Android, SQL, Assembly
\end{body}

\smallskip

% \newpage{} % uncomment this line if you want to force a new page


%%%%%%%%%%%%%%%%%%%%%%%%%%%%%%%%%%%%%%%%%%%%%%%%%%%%%%%%%%%%%%%%%%%%%%%%%%%%%%%%
% Work
\header{Work Experience}

\begin{body}
  \vspace{14pt}

  \textbf{Undergraduate Student Researcher}, \emph{UC Berkeley} \hfill \emph{Spring 2014 -- Fall 2014}\\
  \vspace*{-4pt}
  \begin{itemize} \itemsep -0pt  % reduce space between items
  \item I work with Professor Ras Bodik on the synthesis of a layout engine for
    an experimental browser, Servo. I helped built a backend which generates a
    layout engine in Rust, which replaces the hand-written layout engine in
    Servo. I am also working on writing a synthesis algorithm for incremental
    layout schedules, implemented in Rosette, a domain specific language for
    interfacing with SAT/SMT solvers.
  \end{itemize}

  \textbf{Undergraduate Student Instructor for CS61A}, \emph{UC Berkeley} \hfill \emph{Fall 2013 -- Present}\\
  \vspace*{-4pt}
  \begin{itemize} \itemsep -0pt  % reduce space between items
  \item Teach sections and labs, holds office hours
  \item Help write the autograder for projects
  \item Wrote Javascript interpreters for Scheme and Logic languages used in the
    class, so that students can interpret code on their browsers without
    installing interpreters on their machines.
  \item Ran and maintained the codereview system used to give students
    composition feedback from readers
  \end{itemize}

  \textbf{Software Engineering Intern}, \emph{Clover} \hfill \emph{July--August 2013}\\
  \vspace*{-4pt}
  \begin{itemize} \itemsep -0pt  % reduce space between items
  \item Helped improve internal tools
  \item Built an API auto-documentation system; designed and build an API     Explorer: \url{https://www.clover.com/api_explorer}
  \item Created demo app using Clover's API:     \url{https://github.com/clover/example-server}
  \end{itemize}

%  \textbf{Reader for CS61A}, \emph{UC Berkeley} \hfill \emph{Spring 2013}\\
%  \vspace*{-4pt}
%  \begin{itemize} \itemsep -0pt  % reduce space between items
%  \item I provided feedback and comments for students' code and held debugging     sessions
%  \end{itemize}

  \textbf{Math Competition Trainer}, \emph{National University of Singapore High School} \hfill \emph{March 2012}\\
  \vspace*{-4pt}
  \begin{itemize} \itemsep -0pt  % reduce space between items
  \item Compile problems and create training notes
  \item Conduct classes for grade 8-10 students
  \end{itemize}

  \textbf{Physics Competition Trainer}, \emph{National University of Singapore High School} \hfill \emph{March-August 2012}\\
  \vspace*{-4pt}
  \begin{itemize} \itemsep -0pt  % reduce space between items
  \item Prepare PhD-qualifying exam level problems
  \item Conduct classes for grade 11 students
  \item Create and grade a test
  \end{itemize}
\end{body}

\smallskip

%%%%%%%%%%%%%%%%%%%%%%%%%%%%%%%%%%%%%%%%%%%%%%%%%%%%%%%%%%%%%%%%%%%%%%%%%%%%%%%%
% Projects
\header{Selected Projects}

\begin{body}
  \vspace{14pt}

  \textbf{Facebook Group Archiver} \hfill \url{http://archiver.chenyang.co}\\
  A tool for saving Facebook groups in a local database and doing comprehensive searches locally. After the first download, it will sync the local database with the Facebook group during each run. Also includes a web-interface for stats, searching and doing database queries. \\
  \medskip

  \textbf{Interactive SICP Textbook} \hfill \url{http://xuanji.appspot.com/isicp/1-1-elements.html}\\
  Made an interactive version of the classic Structure and Interpretation of Computer Programs book with my friend. I created the asynchronous Javascript-based Scheme interpreter used on the website.\\
  \medskip

  \textbf{WebGL Particle Simulator} \hfill \url{http://www.chenyang.co/particles}\\
  A simulation with thousands of particles attracted by gravity, created with  WebGL and Javascript. \\
  \medskip

  \textbf{Python Control Flow Visualizer} \hfill \url{http://pyvisualizer.chenyang.co}\\
  An online tool that run python programs and visualize the code branching using D3.js\\
  \medskip

  \textbf{Scheme on TI-89} \hfill \url{https://github.com/yuanchenyang/TI89-Scheme}\\
  Built a Scheme interpreter from scratch that runs on my TI-89 graphing calculator. It is written in C and supports a small subset of the Scheme language.\\
  \medskip

  \textbf{Building a Computer from Scratch} \hfill \url{https://github.com/yuanchenyang/My-EOCS}\\
  Following the instructions from a book called the Elements of Computing Systems, I built a CPU from logic gates using a hardware simulator. Then I proceeded to create an assembler for the CPU and a VM simulator that takes in VM code (similar to java bytecode) and outputs assembly code. \\
  \medskip

 \textbf{Logic Gate Simulator} \hfill \url{https://github.com/yuanchenyang/Logic-Simulator}\\
 Used Python to create a logic gate simulation system with constraint passing. This system also allows powerful abstractions to be made so that more complicated sets of gates can be created, saved and reused. This project won an honorable mention in the Facebook Battle of the Bay hackathon.\\
   \medskip

   \textbf{Perfect Strategy for Hog} \hfill \url{https://github.com/yuanchenyang/Hog-Perfect-Strategy}\\
   For a project in my CS class, we have to create artificial intelligence agents to compete in a dice game called Hog. I used dynamic programming and recursion to create a prefect strategy that cannot be beaten, thereby winning the contest. \\
   \medskip
\end{body}

\smallskip

% \newpage{} % uncomment this line if you want to force a new page

%%%%%%%%%%%%%%%%%%%%%%%%%%%%%%%%%%%%%%%%%%%%%%%%%%%%%%%%%%%%%%%%%%%%%%%%%%%%%%%%
% Awards and Honors
\header{Relevant Awards}

\begin{body}
  \vspace{14pt}
  \textbf{First Place}, Cal vs Stanford Big Hack \hfill{} \emph{Apr 2013}\\
  Created a scheme interpreter in C on my TI-89 graphing calculator \\
  \textbf{Third Place}, Hackers at Berkeley HackJam \hfill{} \emph{Apr 2013}\\
  Made an animation sequence on my TI-89 graphing calculator \\
  \textbf{Honorable Mention}, Facebook Nor-Cal Hackathon 2013 \hfill{} \emph{Oct 2013}\\
  Built a online Python code branching visualizer.\\
  \textbf{Honorable Mention}, Facebook Battle of the Bay Hackathon 2012 \hfill{} \emph{Oct 2012}\\
  Build a logic gate simulator with a graphical interface in Python. \\
  \textbf{Rank 15}, Hackerrank Back to School Hackathon 2013 \hfill{} \emph{Feb 2013}\\
\end{body}

\smallskip


%%%%%%%%%%%%%%%%%%%%%%%%%%%%%%%%%%%%%%%%%%%%%%%%%%%%%%%%%%%%%%%%%%%%%%%%%%%%%%%%
% Coursework
% \header{Relevant Coursework}

% \begin{body}
%   \vspace{14pt}
%   \textbf{CS61A}, Structure and Interpretation of Computer Programs \hfill {} \emph{Fall 2012}\\
%   \medskip
%   \textbf{CS61B}, Data Structures and Algorithms \hfill {} \emph{Spring 2013}\\
%   \medskip
%   \textbf{EECS70}, Discrete Mathematics and Probability Theory \hfill {} \emph{Spring 2013}
% \end{body}
\end{document}
