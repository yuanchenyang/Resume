\documentclass[9pt]{article}
\usepackage{fullpage}
\usepackage{amsmath}
\usepackage{amssymb}
\usepackage[usenames]{color}
\usepackage{hyperref}
\usepackage{multicol}

\leftmargin=0.25in
\oddsidemargin=0.25in
\textwidth=6.0in
\topmargin=-0.25in
\textheight=9.25in

\raggedright

\pagenumbering{arabic}

\def\bull{\vrule height 0.8ex width .7ex depth -.1ex }
% DEFINITIONS FOR RESUME

\newenvironment{changemargin}[2]{%
  \begin{list}{}{%
      \setlength{\topsep}{0pt}%
      \setlength{\leftmargin}{#1}%
      \setlength{\rightmargin}{#2}%
      \setlength{\listparindent}{\parindent}%
      \setlength{\itemindent}{\parindent}%
      \setlength{\parsep}{\parskip}%
    }%
  \item[]}{\end{list}
}

\newcommand{\lineover}{
  \begin{changemargin}{-0.05in}{-0.05in}
    \vspace*{-8pt}
    \hrulefill \\
    \vspace*{-2pt}
  \end{changemargin}
}

\newcommand{\header}[1]{
  \begin{changemargin}{-0.5in}{-0.5in}
    \scshape{#1}\\
    \lineover
  \end{changemargin}
}

\newcommand{\contact}[2]{
  \begin{changemargin}{-0.5in}{-0.5in}
    \begin{center}
      {\Large \scshape {#1}}\\ \smallskip
      {#2}\\ \smallskip
    \end{center}
  \end{changemargin}
}

\newenvironment{body} {
  \vspace*{-16pt}
  \begin{changemargin}{-0.25in}{-0.5in}
  }
  {\end{changemargin}
}

\newcommand{\school}[4]{
  \textbf{#1} \hfill \emph{#2\\}
  #3\\
  #4\\
}

% END RESUME DEFINITIONS

\begin{document}

%%%%%%%%%%%%%%%%%%%%%%%%%%%%%%%%%%%%%%%%%%%%%%%%%%%%%%%%%%%%%%%%%%%%%%%%%%%%%%%%
% Name
\contact{Chenyang Yuan}{\url{yuanchenyang@gmail.com} \quad
  \url{http://www.chenyang.co} \quad \url{http://www.github.com/yuanchenyang}
}
%%%%%%%%%%%%%%%%%%%%%%%%%%%%%%%%%%%%%%%%%%%%%%%%%%%%%%%%%%%%%%%%%%%%%%%%%%%%%%%%
% Work
\header{Work}
\begin{body}
  \vspace{14pt}
  \textbf{Research Scientist } \hfill  2022--Present\\
  \emph{Toyota Research Institute}, Cambridge, MA  \\
\end{body}

%%%%%%%%%%%%%%%%%%%%%%%%%%%%%%%%%%%%%%%%%%%%%%%%%%%%%%%%%%%%%%%%%%%%%%%%%%%%%%%%
% Education
\header{Education}
\begin{body}
  \vspace{14pt}
  \textbf{PhD in Electrical Engineering and Computer Science } \hfill  2018 -- 2022\\
  \emph{Massachusetts Institute of Technology}, Cambridge, MA{} \hfill GPA: 5/5 \\
  Thesis: Polynomial Structure in Semidefinite Relaxations and Non-Convex Formulations \\

  \vspace{5pt}
  \textbf{MS in Electrical Engineering and Computer Science } \hfill  2016 -- 2018 \\
  \emph{Massachusetts Institute of Technology}, Cambridge, MA{} \hfill GPA: 5/5 \\
  Thesis: Focused Polynomials, Random Projections and Approximation Algorithms
  for Polynomial Optimization over the Sphere \\

  \vspace{5pt}
  \textbf{BA in Computer Science} \hfill  2012--2016 \\
  \emph{The University of Berkeley at California}, Berkeley, CA{} \hfill GPA: 3.94/4 \\
\end{body}
\smallskip

%%%%%%%%%%%%%%%%%%%%%%%%%%%%%%%%%%%%%%%%%%%%%%%%%%%%%%%%%%%%%%%%%%%%%%%%%%%%%%%%
\header{Research}
\begin{body}
  \vspace{14pt}

  Beno\^it Legat*, \textbf{Chenyang Yuan}* and Pablo Parrilo, ``Low Rank
  Univariate Sum of
  Squares Has No Spurious Local Minima'', \emph{Preprint, arxiv}\\
  \smallskip

  \textbf{Chenyang Yuan} and Pablo Parrilo,
  ``Rounding Semidefinite Relaxations of Quadratic Maps'', \emph{In preparation}\\
  \smallskip

  \textbf{Chenyang Yuan} and Pablo Parrilo, ``Semidefinite Relaxations of
  Products of
  Nonnegative Forms on the Sphere'', \emph{Preprint, arxiv}\\
  \smallskip

  \textbf{Chenyang Yuan} and Pablo Parrilo, ``Maximizing Products of Linear
  Forms, and
  the Permanent of Positive Semidefinite Matrices'', \emph{Mathematical Programming Series A}\\
  \smallskip

  J. Thai, \textbf{C. Yuan}, A. Bayen, ``Resiliency of Mobility-as-a-Service
  Systems to
  Denial-of-Service Attacks'', \emph{IEEE Transactions on Control of Network Systems}\\
  \smallskip

  \textbf{C. Yuan}, J. Thai, A. Bayen, ``ZUbers against ZLyfts Apocalypse: An
  Analysis Framework for DoS Attacks on Mobility-as-a-Service Systems'',
  \emph{ACM/IEEE International Conference on Cyber-Physical Systems
    (ICCPS)}
\end{body}
\smallskip

%%%%%%%%%%%%%%%%%%%%%%%%%%%%%%%%%%%%%%%%%%%%%%%%%%%%%%%%%%%%%%%%%%%%%%%%%%%%%%%%
% \header{Thesis}
% \begin{body}
%   \vspace{14pt}
%   Chenyang Yuan, ``Focused Polynomials, Random Projections and Approximation Algorithms for
%   Polynomial Optimization over the Sphere'' \emph{SM Thesis, MIT, 2018}
% \end{body}
%
% \smallskip

%%%%%%%%%%%%%%%%%%%%%%%%%%%%%%%%%%%%%%%%%%%%%%%%%%%%%%%%%%%%%%%%%%%%%%%%%%%%%%%%
% Work
\header{Internships}
\begin{body}
  \vspace{14pt}
  \textbf{Research Intern}, \emph{Lyft Inc.} \hfill \emph{June -- September 2016}\\
  \vspace*{-5pt}
  \begin{itemize} \itemsep -0pt  % reduce space between items
  \item Worked with locations team on estimation of travel times using real-time
    traffic data derived from driver GPS routes.
  \end{itemize}

  \textbf{Undergraduate Student Researcher}, \emph{UC Berkeley} \hfill \emph{Spring 2015 -- Spring 2016}\\
  \vspace*{-4pt}
  \begin{itemize} \itemsep -0pt  % reduce space between items
  \item With professor Alex Bayen's group, worked on applying optimization to
    traffic control, inferring route flows of cars from cellular connection data
    and using queueing theory to investigate possible attacks on on-demand
    rideshare networks.
  \end{itemize}

  \textbf{Undergraduate Student Researcher}, \emph{UC Berkeley} \hfill \emph{Spring 2014 -- Spring 2015}\\
  \vspace*{-4pt}
  \begin{itemize} \itemsep -0pt  % reduce space between items
  \item With professor Ras Bodik's group on the synthesis of a layout engine
    for an experimental browser, Servo, using SAT/SMT solvers.
    % I helped built a backend which generates
    % a layout engine in Rust, which replaces the hand-written layout engine in
    % Servo. I also worked on writing a synthesis algorithm for incremental layout
    % schedules, implemented in Rosette, a domain specific language for
    % interfacing with SAT/SMT solvers.
  \end{itemize}

  \textbf{Software Engineering Intern}, \emph{Clover Network Inc.} \hfill \emph{June -- September 2013}\\
  \vspace*{-5pt}
  \begin{itemize} \itemsep -0pt  % reduce space between items
  \item Amongst other projects, designed and built an API auto-documentation system and API Explorer.
  \end{itemize}
\end{body}
\smallskip

%%%%%%%%%%%%%%%%%%%%%%%%%%%%%%%%%%%%%%%%%%%%%%%%%%%%%%%%%%%%%%%%%%%%%%%%%%%%%%%%
% Teaching
%\header{Teaching}
%\begin{body}
%  \vspace{14pt}
%  \textbf{Teaching Assistant}, \emph{MIT} \hfill \emph{Spring 2020 -- Fall 2021}\\
%  \vspace*{-5pt}
%  \begin{itemize} \itemsep -0pt  % reduce space between items
%  \item TA for Nonlinear Optimization (grad), Linear Algebra and Optimization
%    (undergrad), Algebraic Techniques and Semidefinite Programming (grad)
%  \end{itemize}
%
%  \textbf{Undergraduate Student Instructor}, \emph{UC Berkeley} \hfill \emph{Fall 2013 -- Spring 2016}\\
%  \vspace*{-5pt}
%  \begin{itemize} \itemsep -0pt  % reduce space between items
%  \item Structure and Interpretation of Computer Programs, Discrete Math and
%    Probability, Designing Information Devices and Systems, Efficient Algorithms
%    and Intractable Problems.
%  \item I worked with Professor Ras Bodik on the synthesis of a layout engine
%    for an experimental browser, Servo. I helped built a backend which generates
%    a layout engine in Rust, which replaces the hand-written layout engine in
%    Servo. I also worked on writing a synthesis algorithm for incremental layout
%    schedules, implemented in Rosette, a domain specific language for
%    interfacing with SAT/SMT solvers.
%  \end{itemize}
%
%  \textbf{Reader for CS61A}, \emph{UC Berkeley} \hfill \emph{Spring 2013}\\
%  \vspace*{-4pt}
%  \begin{itemize} \itemsep -0pt  % reduce space between items
%  \item I provided feedback and comments for students' code and held debugging     sessions
%  \end{itemize}
%\end{body}

%\smallskip

%%%%%%%%%%%%%%%%%%%%%%%%%%%%%%%%%%%%%%%%%%%%%%%%%%%%%%%%%%%%%%%%%%%%%%%%%%%%%%%%
% Skills
\header{Programming Skills}

\begin{body}
  \vspace{14pt}
  \textbf{Proficient in} Python, Julia, PyTorch, Javascript, \LaTeX, Emacs, Git, Docker \\
  \smallskip
  \textbf{Experience in} Java, C, Rust, Haskell, Scheme, HTML/CSS, Android, SQL,
  Assembly
\end{body}

\smallskip


%%%%%%%%%%%%%%%%%%%%%%%%%%%%%%%%%%%%%%%%%%%%%%%%%%%%%%%%%%%%%%%%%%%%%%%%%%%%%%%%
% Talks
\header{Talks}
\begin{body}
  \vspace{14pt}
  \textbf{ICCOPT} Invited talk in Session on Algorithms for Large-scale Conic and Polynomial Optimization \hfill{} \emph{Jul 2022}\\

  \textbf{MIT} LIDS and Stats Tea Talk \hfill{} \emph{Dec 2021}\\
  \smallskip

  \textbf{INFORMS Annual Meeting} Optimization in Julia Session \hfill{} \emph{Oct 2021} \\
  \smallskip

  \textbf{Fields Institute} Workshop on Real Algebraic Geometry and Algorithms
   \hfill{} \emph{Jun 2021}\\
  \smallskip

  \textbf{MIT} LIDS Student Conference \hfill{} \emph{Jan 2021}\\
  \smallskip

  \textbf{MIT} CS Theory Lunch \hfill{} \emph{Feb 2020}\\
  \smallskip
\end{body}

\smallskip


%%%%%%%%%%%%%%%%%%%%%%%%%%%%%%%%%%%%%%%%%%%%%%%%%%%%%%%%%%%%%%%%%%%%%%%%%%%%%%%%
% Teaching

\header{Teaching }

\begin{body}
  \vspace{14pt}

  \textbf{Algebraic Techniques and Semidefinite Programming}, \emph{MIT}
  \hfill \emph{Spring 2021} \\
  \smallskip

  \textbf{Linear Algebra and Optimization}, \emph{MIT} \hfill \emph{Fall 2020/2021} \\
  \smallskip

  %\vspace*{-4pt}
  %\begin{itemize} \itemsep -0pt  % reduce space between items
  %\item Undergrad class designed to emphasize linear algebra
  %\end{itemize}

  \textbf{Nonlinear Optimization}, \emph{MIT} \hfill \emph{Spring 2020}\\
  \smallskip

  \textbf{Efficient Algorithms and Intractable Problems}, \emph{UC Berkeley} \hfill \emph{Spring 2016}\\
  \smallskip

  \textbf{Designing Information Devices and Systems}, \emph{UC Berkeley} \hfill \emph{Fall 2015}\\
  \smallskip
  % \vspace*{-4pt}
  % \begin{itemize} \itemsep -0pt  % reduce space between items
  % \item New class designed to introduce linear algebra and applications to first
  %   year students
  % \item Helped create labs, homework questions and write class notes
  % \end{itemize}

  \textbf{TA for Structure and Interpretation of Computer Programs}, \emph{UC Berkeley} \hfill \emph{Fall 2013 -- Fall 2014}\\
  \smallskip
  %\vspace*{-4pt}
  %\begin{itemize} \itemsep -0pt  % reduce space between items
  %\item Teach sections and labs, holds office hours
  %\item Help write the autograder for projects
  %\item Wrote Javascript interpreters for Scheme and Logic languages used in the
  %  class, so that students can interpret code on their browsers without
  %  installing interpreters on their machines.
  %\item Ran and maintained the codereview system used to give students
  %  composition feedback from readers
  %\end{itemize}

  \textbf{Math Olympiad Trainer}, \emph{National University of Singapore High School} \hfill \emph{March 2012}\\
  \smallskip
  % \vspace*{-4pt}
  % \begin{itemize} \itemsep -0pt  % reduce space between items
  % \item Compile problems and create training notes
  % \item Conduct classes for grade 8-10 students
  % \end{itemize}

  \textbf{Physics Olympiad Trainer}, \emph{National University of Singapore High School} \hfill \emph{March-August 2012}\\
  % \vspace*{-4pt}
  % \begin{itemize} \itemsep -0pt  % reduce space between items
  % \item Prepare PhD-qualifying exam level problems
  % \item Conduct classes for grade 11 students
  % \item Create and grade a test
  % \end{itemize}
\end{body}
\smallskip

%%%%%%%%%%%%%%%%%%%%%%%%%%%%%%%%%%%%%%%%%%%%%%%%%%%%%%%%%%%%%%%%%%%%%%%%%%%%%%%%
% Projects
\header{Selected Software Projects}
\begin{body}
  \vspace{14pt}
  \textbf{SumOfSquares.py} \hfill \url{https://github.com/yuanchenyang/SumOfSquares.py} \\
  Sum of squares optimization modeller built on top of picos. Features easy
  access to pseudoexpectation operators for both formulating problems and
  extracting solutions via rounding algorithms
  \smallskip

%   \textbf{Linear Algrbra DSL} \hfill \url{https://github.com/yuanchenyang/llvm-linear-algebra-dsl} \\
%   An open-ended project for a compilers class. First created a set of tools for
%   building domain specific languages (DSLs) using LLVM for code generation and
%   created a DSL for linear algebra operations introducing lots of
%   domain-specific optimizations. Then implemented an edge detector and part of
%   an optical flow estimation algorithm using the DSL.  \medskip
%
%   \textbf{Facebook Group Archiver} \hfill \url{http://archiver.chenyang.co}\\
%   A tool for saving Facebook groups in a local database and doing comprehensive
%   searches locally. After the first download, it will sync the local database
%   with the Facebook group during each run. Also includes a web-interface for
%   stats, searching and doing database queries. \\
%   \medskip

  \textbf{Interactive SICP Textbook / coding.js} \hfill
  \url{http://xuanji.appspot.com/isicp/1-1-elements.html}\\
  An interactive version of the classic Structure and Interpretation of Computer
  Programs book, created together with a friend. I wrote the asynchronous
  Javascript-based Scheme interpreter used on the website. \\
\end{body}

\smallskip

% \newpage{} % uncomment this line if you want to force a new page

%%%%%%%%%%%%%%%%%%%%%%%%%%%%%%%%%%%%%%%%%%%%%%%%%%%%%%%%%%%%%%%%%%%%%%%%%%%%%%%%
% Awards and Honors
\header{Selected Awards}
\begin{body}
  \vspace{14pt}
  \textbf{Outstanding Course Development and Teaching Award}, for developing
  a new linear algebra course (EE16A) at UC Berkeley \hfill{} \emph{May 2016}\\
  \smallskip
  \textbf{First Place}, Cal vs Stanford Big Hack \hfill{} \emph{Apr 2013}\\
  Created a scheme interpreter in C on my TI-89 graphing calculator \\
  \smallskip
  \textbf{Honorable Mention}, 12th Asian Physics Olympiad \hfill{} \emph{May 2011}\\
  One of the 8 students representing Singapore in this competition.\\
  % \textbf{Third Place}, Hackers at Berkeley HackJam \hfill{} \emph{Apr 2013}\\
  % Made an animation sequence on my TI-89 graphing calculator \\
  % \textbf{Honorable Mention}, Facebook Nor-Cal Hackathon 2013 \hfill{} \emph{Oct 2013}\\
  % Built a online Python code branching visualizer.\\
  % \textbf{Honorable Mention}, Facebook Battle of the Bay Hackathon 2012 \hfill{} \emph{Oct 2012}\\
  % Build a logic gate simulator with a graphical interface in Python. \\
  % \textbf{Rank 15}, Hackerrank Back to School Hackathon 2013 \hfill{} \emph{Feb 2013}\\
\end{body}

\smallskip

%%%%%%%%%%%%%%%%%%%%%%%%%%%%%%%%%%%%%%%%%%%%%%%%%%%%%%%%%%%%%%%%%%%%%%%%%%%%%%%%
% Reviewing
\header{Reviewing Experience}
\begin{body}
  \vspace{14pt} Optimization Letters; Journal of Combinatorics; International
  Colloquium on Automata, Languages, and Programming (ICALP); Sum of Squares:
  Theory and Applications (book chapter)
\end{body}
\smallskip

%%%%%%%%%%%%%%%%%%%%%%%%%%%%%%%%%%%%%%%%%%%%%%%%%%%%%%%%%%%%%%%%%%%%%%%%%%%%%%%%
% Coursework
\header{Selected Coursework}
\begin{body}
  \vspace{14pt}
  \textbf{CS:} \emph{Berkeley:} Graduate Algorithms and Theory, Compilers, Security, AI, Randomized Algorithms. \emph{MIT:} Advanced Algorithms, Inference and Information, Geometric Computing, Algebraic Techniques and Semidefinite Programming \\
  \smallskip
  \textbf{EE:} \emph{Berkeley:} Information Theory, \emph{MIT:} Dynamic Systems and Control \\
  \smallskip
  \textbf{Math:} \emph{Berkeley:} Complex Analysis, Honors Abstract Algebra. \emph{MIT:} High-dimensional Statistics \\
  \smallskip
  \textbf{Physics:} \emph{Berkeley:} Analytical Mechanics, Quantum Mechanics, General Relativity, Electronics Lab \\
\end{body}
\end{document}
