\documentclass[9pt]{article}
\usepackage{fullpage}
\usepackage{amsmath}
\usepackage{amssymb}
\usepackage[usenames]{color}
\usepackage{hyperref}
\usepackage{multicol}

\leftmargin=0.25in
\oddsidemargin=0.25in
\textwidth=6.0in
\topmargin=-0.25in
\textheight=9.25in

\raggedright

\pagenumbering{arabic}

\def\bull{\vrule height 0.8ex width .7ex depth -.1ex }
% DEFINITIONS FOR RESUME

\newenvironment{changemargin}[2]{%
  \begin{list}{}{%
      \setlength{\topsep}{0pt}%
      \setlength{\leftmargin}{#1}%
      \setlength{\rightmargin}{#2}%
      \setlength{\listparindent}{\parindent}%
      \setlength{\itemindent}{\parindent}%
      \setlength{\parsep}{\parskip}%
    }%
  \item[]}{\end{list}
}

\newcommand{\lineover}{
  \begin{changemargin}{-0.05in}{-0.05in}
    \vspace*{-8pt}
    \hrulefill \\
    \vspace*{-2pt}
  \end{changemargin}
}

\newcommand{\header}[1]{
  \begin{changemargin}{-0.5in}{-0.5in}
    \scshape{#1}\\
    \lineover
  \end{changemargin}
}

\newcommand{\contact}[4]{
  \begin{changemargin}{-0.5in}{-0.5in}
    \begin{center}
      {\Large \scshape {#1}}\\ \smallskip
      {#2}\\ \smallskip
      {#3}\\ \smallskip
      {#4}\smallskip
    \end{center}
  \end{changemargin}
}

\newenvironment{body} {
  \vspace*{-16pt}
  \begin{changemargin}{-0.25in}{-0.5in}
  }
  {\end{changemargin}
}

\newcommand{\school}[4]{
  \textbf{#1} \hfill \emph{#2\\}
  #3\\
  #4\\
}

% END RESUME DEFINITIONS

\begin{document}

%%%%%%%%%%%%%%%%%%%%%%%%%%%%%%%%%%%%%%%%%%%%%%%%%%%%%%%%%%%%%%%%%%%%%%%%%%%%%%%%
% Name
\contact{Chenyang Yuan}{\href{mailto:yuanchenyang@gmail.com}{yuanchenyang@gmail.com}}{ \url{http://www.github.com/yuanchenyang} \quad \url{http://www.chenyang.co}}

%%%%%%%%%%%%%%%%%%%%%%%%%%%%%%%%%%%%%%%%%%%%%%%%%%%%%%%%%%%%%%%%%%%%%%%%%%%%%%%%
% Education
\header{Education}

\begin{body}
  \vspace{14pt}
  \textbf{PhD in Electrical Engineering and Computer Science } \hfill  2018--Present\\
  \emph{Massachusetts Institute of Technology}, Cambridge, MA{} \hfill GPA: 5/5 \\

  \vspace{5pt}
  \textbf{MA in EECS } \hfill  2016--2018 \\
  \emph{Massachusetts Institute of Technology}, Cambridge, MA{} \hfill GPA: 5/5 \\

  \vspace{5pt}
  \textbf{BA in Computer Science} \hfill  2012--2016 \\
  \emph{The University of Berkeley at California}, Berkeley, CA{} \hfill GPA: 3.94/4 \\
\end{body}

\smallskip

%%%%%%%%%%%%%%%%%%%%%%%%%%%%%%%%%%%%%%%%%%%%%%%%%%%%%%%%%%%%%%%%%%%%%%%%%%%%%%%%
% Papers
\header{Papers}
\begin{body}
  \vspace{14pt}
  \textbf{Chenyang Yuan} and Pablo Parrilo, ``Semidefinite Relaxations of Products of
  Nonnegative Forms on the Sphere'', \emph{Preprint, arxiv}\\
  \textbf{Chenyang Yuan} and Pablo Parrilo, ``Maximizing Products of Linear Forms, and
  the Permanent of Positive Semidefinite Matrices'', \emph{Mathematical Programming Series A}\\
  J. Thai, \textbf{C. Yuan}, A. Bayen, ``Resiliency of Mobility-as-a-Service Systems to
  Denial-of-Service Attacks'', \emph{2016 IEEE Transactions on Control of Network Systems}\\
  \textbf{C. Yuan}, J. Thai, A. Bayen, ``ZUbers against ZLyfts Apocalypse: An Analysis
  Framework for DoS Attacks on Mobility-as-a-Service Systems'',
  \emph{2016 ACM/IEEE International Conference on Cyber-Physical Systems (ICCPS)}
\end{body}
\smallskip

%%%%%%%%%%%%%%%%%%%%%%%%%%%%%%%%%%%%%%%%%%%%%%%%%%%%%%%%%%%%%%%%%%%%%%%%%%%%%%%%
% Papers
\header{Thesis}

\begin{body}
  \vspace{14pt}
  Chenyang Yuan, ``Focused Polynomials, Random Projections and Approximation Algorithms for
  Polynomial Optimization over the Sphere'' \emph{SM Thesis, MIT, 2018}
\end{body}

%%%%%%%%%%%%%%%%%%%%%%%%%%%%%%%%%%%%%%%%%%%%%%%%%%%%%%%%%%%%%%%%%%%%%%%%%%%%%%%%
% Skills
\header{Programming Skills}

\begin{body}
  \vspace{14pt}
  \textbf{Proficient in} Python, Julia, Javascript, \LaTeX, Emacs, Git\\
  \textbf{Experience in} Java, C, Rust, Haskell, Scheme, HTML, Hadoop, Android, SQL, Assembly
\end{body}

\smallskip

% \newpage{} % uncomment this line if you want to force a new page


%%%%%%%%%%%%%%%%%%%%%%%%%%%%%%%%%%%%%%%%%%%%%%%%%%%%%%%%%%%%%%%%%%%%%%%%%%%%%%%%
% Work
\header{Work Experience}

\begin{body}
  \vspace{14pt}

  \textbf{Research Intern}, \emph{Lyft} \hfill \emph{--August 2013}\\
  \vspace*{-4pt}
  \begin{itemize} \itemsep -0pt  % reduce space between items
  \item Helped improve internal tools
  \item Built an API auto-documentation system; designed and build an API     Explorer: \url{https://www.clover.com/api_explorer}
  \item Created demo app using Clover's API:     \url{https://github.com/clover/example-server}
  \end{itemize}

  \textbf{Undergraduate Student Researcher}, \emph{UC Berkeley} \hfill \emph{Spring 2015 -- Spring 2016}\\
  \vspace*{-4pt}
  \begin{itemize} \itemsep -0pt  % reduce space between items
  \item I work on traffic research with Professor Alex Bayen. Projects I worked
    on include inferring route flows of cars from cellular connection data and
    using queueing theory to investigate possible attacks on on-demand taxi
    networks by calling taxies and then canceling the calls.
  \end{itemize}

  \textbf{Undergraduate Student Researcher}, \emph{UC Berkeley} \hfill \emph{Spring 2014 -- Fall 2014}\\
  \vspace*{-4pt}
  \begin{itemize} \itemsep -0pt  % reduce space between items
  \item I worked with Professor Ras Bodik on the synthesis of a layout engine
    for an experimental browser, Servo. I helped built a backend which generates
    a layout engine in Rust, which replaces the hand-written layout engine in
    Servo. I also worked on writing a synthesis algorithm for incremental layout
    schedules, implemented in Rosette, a domain specific language for
    interfacing with SAT/SMT solvers.
  \end{itemize}

  \textbf{Software Engineering Intern}, \emph{Clover} \hfill \emph{June--September 2013}\\
  \vspace*{-4pt}
  \begin{itemize} \itemsep -0pt  % reduce space between items
  \item Helped improve internal tools
  \item Built an API auto-documentation system; designed and build an API     Explorer: \url{https://www.clover.com/api_explorer}
  \item Created demo app using Clover's API:     \url{https://github.com/clover/example-server}
  \end{itemize}

%  \textbf{Reader for CS61A}, \emph{UC Berkeley} \hfill \emph{Spring 2013}\\
%  \vspace*{-4pt}
%  \begin{itemize} \itemsep -0pt  % reduce space between items
%  \item I provided feedback and comments for students' code and held debugging     sessions
%  \end{itemize}
\end{body}

\smallskip

%%%%%%%%%%%%%%%%%%%%%%%%%%%%%%%%%%%%%%%%%%%%%%%%%%%%%%%%%%%%%%%%%%%%%%%%%%%%%%%%
% Teaching

\header{Teaching Experience}

\begin{body}
  \vspace{14pt}

  \textbf{TA for Algebraic Techniques and Semidefinite Programming}, \emph{MIT} \hfill \emph{Spring 2021}

  \textbf{TA for Linear Algebra and Optimization}, \emph{MIT} \hfill \emph{Fall 2020}
  \vspace*{-4pt}
  \begin{itemize} \itemsep -0pt  % reduce space between items
  \item Undergrad class designed to emphasize linear algebra
  \item
  \end{itemize}

  \textbf{TA for Nonlinear Optimization}, \emph{MIT} \hfill \emph{Spring 2020}

  \textbf{TA for Designing Information Devices and Systems}, \emph{UC Berkeley} \hfill \emph{Fall 2015}\\
  \vspace*{-4pt}
  \begin{itemize} \itemsep -0pt  % reduce space between items
  \item New class designed to introduce linear algebra and applications to first
    year students
  \item Helped create labs, homework questions and write class notes
  \end{itemize}

  \textbf{TA for Discrete Math and Probability}, \emph{UC Berkeley} \hfill \emph{Spring 2015}\\
  \vspace*{-4pt}
  \begin{itemize} \itemsep -0pt  % reduce space between items
  \item Teach sections and labs, holds office hours
  \item On the content team that creates weekly homeworks, discussion sheets and their solutions
  \item Wrote an interactive browser-based virtual lab for polynomial interpolation
  \end{itemize}

  \textbf{TA for Structure and Interpretation of Computer Programs}, \emph{UC Berkeley} \hfill \emph{Fall 2013 -- Fall 2014}\\
  \vspace*{-4pt}
  \begin{itemize} \itemsep -0pt  % reduce space between items
  \item Teach sections and labs, holds office hours
  \item Help write the autograder for projects
  \item Wrote Javascript interpreters for Scheme and Logic languages used in the
    class, so that students can interpret code on their browsers without
    installing interpreters on their machines.
  \item Ran and maintained the codereview system used to give students
    composition feedback from readers
  \end{itemize}

  \textbf{Math Competition Trainer}, \emph{National University of Singapore High School} \hfill \emph{March 2012}\\
  \vspace*{-4pt}
  \begin{itemize} \itemsep -0pt  % reduce space between items
  \item Compile problems and create training notes
  \item Conduct classes for grade 8-10 students
  \end{itemize}

  \textbf{Physics Competition Trainer}, \emph{National University of Singapore High School} \hfill \emph{March-August 2012}\\
  \vspace*{-4pt}
  \begin{itemize} \itemsep -0pt  % reduce space between items
  \item Prepare PhD-qualifying exam level problems
  \item Conduct classes for grade 11 students
  \item Create and grade a test
  \end{itemize}
\end{body}

%%%%%%%%%%%%%%%%%%%%%%%%%%%%%%%%%%%%%%%%%%%%%%%%%%%%%%%%%%%%%%%%%%%%%%%%%%%%%%%%
% Projects
\header{Software Projects}

\begin{body}
  \vspace{14pt}

  \textbf{SumOfSquares.py} \hfill \url{https://github.com/yuanchenyang/SumOfSquares.py} \\
  Sum of squares optimization modeller built on top of picos. Features easy
  access to pseudoexpectation operators for both formulating problems and
  extracting solutions via rounding algorithms
  \medskip

  \textbf{Linear Algrbra DSL} \hfill \url{https://github.com/yuanchenyang/llvm-linear-algebra-dsl} \\
  An open-ended project for a compilers class. First created a set of tools for
  building domain specific languages (DSLs) using LLVM for code generation and
  created a DSL for linear algebra operations introducing lots of
  domain-specific optimizations. Then implemented an edge detector and part of
  an optical flow estimation algorithm using the DSL.
  \medskip

  \textbf{Facebook Group Archiver} \hfill \url{http://archiver.chenyang.co}\\
  A tool for saving Facebook groups in a local database and doing comprehensive
  searches locally. After the first download, it will sync the local database
  with the Facebook group during each run. Also includes a web-interface for
  stats, searching and doing database queries. \\
  \medskip

  \textbf{Interactive SICP Textbook} \hfill
  \url{http://xuanji.appspot.com/isicp/1-1-elements.html}\\
  Made an interactive version of the classic Structure and Interpretation of
  Computer Programs book with my friend. I created the asynchronous
  Javascript-based Scheme interpreter used on the website. \\
  \medskip

  \textbf{Self-Balancing Robot} \hfill \url{http://youtu.be/Ps0Ex3ADR6k} \\
  An open-ended project for my physics electronics lab class, built a
  self-balancing robot from scratch. Programmed a controller for it on an
  Arduino board. \\
  \medskip

  \textbf{WebGL Particle Simulator} \hfill \url{http://www.chenyang.co/particles}\\
  A simulation with thousands of particles attracted by gravity, created with  WebGL and Javascript. \\
  \medskip

  \textbf{Python Control Flow Visualizer} \hfill \url{http://pyvisualizer.chenyang.co}\\
  An online tool that run python programs and visualize the code branching using D3.js\\
  \medskip

  \textbf{Scheme on TI-89} \hfill \url{https://github.com/yuanchenyang/TI89-Scheme}\\
  Built a Scheme interpreter from scratch that runs on my TI-89 graphing calculator. It is written in C and supports a small subset of the Scheme language.\\
  \medskip

  \textbf{Building a Computer from Scratch} \hfill \url{https://github.com/yuanchenyang/My-EOCS}\\
  Following the instructions from a book called the Elements of Computing Systems, I built a CPU from logic gates using a hardware simulator. Then I proceeded to create an assembler for the CPU and a VM simulator that takes in VM code (similar to java bytecode) and outputs assembly code. \\
  \medskip

 \textbf{Logic Gate Simulator} \hfill \url{https://github.com/yuanchenyang/Logic-Simulator}\\
 Used Python to create a logic gate simulation system with constraint passing. This system also allows powerful abstractions to be made so that more complicated sets of gates can be created, saved and reused. This project won an honorable mention in the Facebook Battle of the Bay hackathon.\\
   \medskip

   \textbf{Perfect Strategy for Hog} \hfill \url{https://github.com/yuanchenyang/Hog-Perfect-Strategy}\\
   For a project in my CS class, we have to create artificial intelligence agents to compete in a dice game called Hog. I used dynamic programming and recursion to create a prefect strategy that cannot be beaten, thereby winning the contest. \\
   \medskip
\end{body}

\smallskip

% \newpage{} % uncomment this line if you want to force a new page

%%%%%%%%%%%%%%%%%%%%%%%%%%%%%%%%%%%%%%%%%%%%%%%%%%%%%%%%%%%%%%%%%%%%%%%%%%%%%%%%
% Awards and Honors
\header{Selected Awards}
\begin{body}
  \vspace{14pt}
  \textbf{First Place}, Cal vs Stanford Big Hack \hfill{} \emph{Apr 2013}\\
  Created a scheme interpreter in C on my TI-89 graphing calculator \\
  \textbf{Honorable Mention}, 12th Asian Physics Olympiad \hfill{} \emph{May 2011}\\
  One of the 8 students representing Singapore in this competition.\\
  \textbf{Third Place}, Hackers at Berkeley HackJam \hfill{} \emph{Apr 2013}\\
  Made an animation sequence on my TI-89 graphing calculator \\
  \textbf{Honorable Mention}, Facebook Nor-Cal Hackathon 2013 \hfill{} \emph{Oct 2013}\\
  Built a online Python code branching visualizer.\\
  \textbf{Honorable Mention}, Facebook Battle of the Bay Hackathon 2012 \hfill{} \emph{Oct 2012}\\
  Build a logic gate simulator with a graphical interface in Python. \\
  \textbf{Rank 15}, Hackerrank Back to School Hackathon 2013 \hfill{} \emph{Feb 2013}\\
\end{body}

\smallskip

%%%%%%%%%%%%%%%%%%%%%%%%%%%%%%%%%%%%%%%%%%%%%%%%%%%%%%%%%%%%%%%%%%%%%%%%%%%%%%%%
% Coursework
\header{Selected Coursework}
\begin{body}
  \vspace{14pt}

  \textbf{CS:} Berkeley: Graduate Algorithms and Theory, Compilers, Security, AI, Randomized Algorithms. MIT: Advanced Algorithms, Inference and Information, Geometric Computing, Algebraic Techniques and Semidefinite Programming \\
  \textbf{EE:} MIT: Dynamic Systems and Control \\
  \medskip
  \textbf{Math:} Berkeley: Complex Analysis, Honors Abstract Algebra. MIT: High-dimensional Statistics \\
  \medskip
  \textbf{Physics:} Berkeley: Analytical Mechanics, Quantum Mechanics, General Relativity, Electronics Lab \\
  \medskip
\end{body}
\end{document}
